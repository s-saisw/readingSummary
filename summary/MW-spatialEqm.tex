\documentclass[a4paper, 12pt]{article}
\usepackage[utf8]{inputenc}
\usepackage[english]{babel}
\usepackage{apacite}
\usepackage{fancyhdr}
\usepackage{geometry}
%\usepackage{hyperref}
\usepackage[hidelinks]{hyperref}
\usepackage{amsmath, amsthm, amssymb, amsfonts}
\usepackage{marginnote}
\geometry{ left=20mm, right=20mm, top=20mm,bottom=20mm, marginparwidth=35mm}
\usepackage{graphicx}
\usepackage{caption}
\usepackage{subcaption}
\usepackage{multicol}
\usepackage{natbib}
\bibliographystyle{apacite}
\graphicspath{ {/school/printscreen/}}

\renewcommand{\vec}[1]{\mathbf{#1}}
\newcommand{\ubar}[1]{\text{\b{$#1$}}}
\linespread{1}

\newcommand{\R}{\mathbb{R}}
\newcommand{\N}{\mathbb{N}}

\newcommand\fnote[1]{\captionsetup{font=small}\caption*{ #1}}

\begin{document}
\pagestyle{fancy}
\setlength{\parindent}{5ex}
\setlength{\columnseprule}{0.5pt}

\title{Summary:
\\Minimum Wages and Spatial Equilibrium\\
\large Journal of Labor Economics, 2019, Vol. 37, no. 3 \\
\large Joan Monras
}
\author{\url{https://github.com/s-saisw/readingSummary}}
\date{\today}
\maketitle

\rhead{Minimum Wages and Spatial Equilibrium}
\lhead{}

\section{Model}
\begin{itemize}
\item There are two regions: Region 1 and Region 2.
\item Each region has a population of $P_i$, where $i \in \{1,2\}$ and $P_1+P_2 = 1$
\item Assumptions
	\begin{itemize}
	\item Amenity levels are equal across regions.
	\item Product demands are not necessarily local.
	\item Home market effects (?)
	\item There is no congestion except from the one coming from the labor market.
	\end{itemize}

\end{itemize}

\subsection{Labor demand}
Firm maximizes
\begin{equation}
\max_{K_i,L_i} AF(K_i, L_i) - r_i K_i - w_i L_i
\end{equation}
Therefore,
\begin{equation}
AF_l(\bar{K}_i, L_i) = w_i.
\end{equation}
It follows that
\begin{equation}
\frac{\partial w_i}{\partial L_i} = \frac{A\partial F_l}{\partial L_i}<0.
\end{equation}
When people move into one region, they exert downward pressure on wage.

\subsection{Mobility decision}
Indirect utility is given by
\begin{equation}
V_i = u_iB_i^\rho + (1-u_i)(1-\tau_i)^\rho w_i^\rho,
\end{equation}
where \\
$u$ : probability of unemployment \\
$B$ : unemployment benefits \\
$\tau$ : tax rate \\
$w$ : wage \\
$\rho$ : risk aversion (when $\rho = 1$, workers are risk neutral)

\subsection{Equilibrium}
Workers are indifferent between living in Region 1 or Region 2. That is 
$$V_1 = V_2.$$

\subsection{Government Budget Constraint}
When unemployment benefits are locally funded,
\begin{equation}
(P_i - L_i)B_i = \tau_iw_iL_i.
\end{equation}
When unemployment benefits are nationally funded,
\begin{equation}
(P_1 - L_1)B_1 + (P_2 - L_2)B_2= \tau_1w_1L_1 + \tau_2w_2L_2.
\end{equation}

\subsection{Equilibrium without Minimum Wages}

There is no unemployment or tax in both regions. Then for any $i$, $V_i = w_i^\rho$. Therefore, at equilibrium,
\begin{align*}
w_1 =& w_2 \\
F_l(\bar{K}_1, L^{FME}_1) =& F_l(\bar{K}_2, L^{FME}_2)
\end{align*}

\subsection{Equilibrium with Minimum Wages}
Introduce minimum wage to Region 1. In Region 2, there is still no unemployment.
\subsubsection{Locally Funded Employment Benefits}

At equilibrium,
$$
u_1B_1^\rho + (1-u_1)(1-\tau_1)^\rho \ubar{w}_1^\rho =
w_2^\rho.
$$

By government budget constraint, $B_1 = \frac{\tau_1w_1L_1}{P_1-L_1}$ and $u_1 = \frac{P_1 - L_1}{P_1}$. Then,

\begin{equation}
\ubar{w}_1^\rho [
u_1^{1-\rho}(1-u_1)^\rho\tau_1^\rho+(1-u_1)(1-\tau_1)^\rho
] =
w_2^\rho.
\label{eq:eqm_local}
\end{equation}

This means expected utility is the minimum wage weighted by the relative employment loss.

\textbf{Proposition 1:} When unemployment benefits are financed locally, there is a threshold value of the labor demand elasticity ($\epsilon_1$) above which Region 1 loses population when minimum wages increase ($\partial P_1/\partial \ubar{w}_1 < 0$).

\textbf{Intuition:} When $\epsilon_1$ is small, employment loss is small and people move into Region 1. On the other hand, when $\epsilon_1$ is large, employment effects do not compensate for higher wage.

\subsubsection{Nationally Funded Employment Benefits}
Unlike the previous case, now workers in Region 2 also pay for unemployment in Region 1 through tax transfers. At equilibrium,
\begin{equation}
u_1B_1^\rho + (1-u_1)(1-\tau_1)^\rho \ubar{w}_1^\rho =
(1-\tau)^\rho w_2^\rho.
\label{eq:eqm_national}
\end{equation}

\textbf{Proposition 2:} When unemployment benefits are financed nationally, Region 1 may gain population if unemployment benefits are sufficiently high, irrespective of the local labor demand elasticity. In general, however, Region 1 gains or loses population depending on the local labor demand elasticity. That is, there is a threshold value of the labor demand elasticity above which Region 1 loses population when minimum wages increase.

\textbf{Intuition:} When unemployment benefits are not zero, there is a transfer from Region 2 to Region 1. If this is high enough, Region 1 can become more attractive. On the other hand, when there is no unemployment benefits, \eqref{eq:eqm_national} reduces to \eqref{eq:eqm_local}.

\section{Empirical Evidence}
\begin{itemize}
\item Data: CPS 1962--2013
\item Sample restriction: High-school graduates
\item Empirical strategy: event study design
\item Independent variables: in-migration, out-migration, low-skilled in-migration rate relative to high-skilled in-migration rate, low-skilled out-migration rate relative to high-skilled out-migration rate
\end{itemize}

\end{document}
