\documentclass[a4paper, 12pt]{article}
\usepackage[utf8]{inputenc}
\usepackage[english]{babel}
\usepackage{apacite}
\usepackage{fancyhdr}
\usepackage{geometry}
%\usepackage{hyperref}
\usepackage[hidelinks]{hyperref}
\usepackage{amsmath, amsthm, amssymb, amsfonts}
\usepackage{bbm}
\usepackage{marginnote}
\geometry{ left=20mm, right=20mm, top=20mm,bottom=20mm, marginparwidth=35mm}
\usepackage{graphicx}
\usepackage{multicol}
\usepackage{natbib}
\bibliographystyle{abbrvnat}
\graphicspath{ {/school/printscreen/}}

\renewcommand{\vec}[1]{\mathbf{#1}}

\linespread{1}

\newcommand{\R}{\mathbb{R}}
\newcommand{\N}{\mathbb{N}}
\newcommand{\E}{\mathbb{E}}
\newcommand{\ind}{\mathbbm{1}}

\newtheorem*{assumption*}{\assumptionnumber}
\providecommand{\assumptionnumber}{}
\makeatletter
\newenvironment{assumption}[2]
 {%
  \renewcommand{\assumptionnumber}{Assumption #1 \textit{(#2)}}%
  \begin{assumption*}%
  \protected@edef\@currentlabel{#1}%
 }
 {%
  \end{assumption*}
 }
\makeatother

\newtheorem*{prop*}{\propnumber}
\providecommand{\propnumber}{}
\makeatletter
\newenvironment{prop}[2]
 {%
  \renewcommand{\propnumber}{Proposition #1 \textit{(#2)}}%
  \begin{prop*}%
  \protected@edef\@currentlabel{#1}%
 }
 {%
  \end{prop*}
 }
\makeatother


\begin{document}
\pagestyle{fancy}
\setlength{\parindent}{5ex}
\setlength{\columnseprule}{0.5pt}

\title{Summary:
\\Chapter 1 Decomposition Methods in Economics \\
\large Nicole Fortin et al.
}
\author{https://github.com/s-saisw/readingSummary}
\date{\today}
\maketitle

\rhead{Decomposition Methods in Economics }
\textbf{Review} \\

\begin{flushright}
Keywords: 
\end{flushright}

\section{Introduction}
\begin{itemize}
\item Decomposition methods are often used to examine differences between two groups e.g. wage gap between union VS non-union members, races, men VS women etc.
\item Limits of decomposition methods:
\begin{itemize}
\item Decomposition methods follow a partial equilibrium approach, ignoring the general equilibrium effect.
\item Decomposition methods do not seek to recover ``deep'' structural parameters.
\end{itemize}
\item Setting: suppose there are 2 groups ($g$), A and B. Assume that 
\begin{enumerate}
\item Outcome variable $Y$ is linearly correlated to covariates $X$.
\item Error term $\nu$ is conditionally dependent of $X$, i.e. $\E(\nu_{gi}|X_i)=0$.
\end{enumerate}
\begin{equation}
Y_{gi}= \beta_{g0} + \sum_{k=1}^K X_{ik} \beta_{gk} + \nu_{gi}, \text{ } g=A,B
\end{equation}
\par Then the average difference between 2 groups, $\hat{\Delta}_O^\mu$, can be written as
\begin{align*}
\hat{\Delta}_O^\mu 
=& 
\hat{\beta}_{B0} + \sum_{k=1}^K \bar{X}_{Bk}\hat{\beta}_{Bk}
- 
(
\hat{\beta}_{A0} + \sum_{k=1}^K \bar{X}_{Ak} \hat{\beta}_{Ak}
) \\
=&
\hat{\beta}_{B0} -\hat{\beta}_{A0}
+ 
\sum_{k=1}^K \bar{X}_{Bk}\hat{\beta}_{Bk}
- 
\sum_{k=1}^K \bar{X}_{Ak} \hat{\beta}_{Ak}
-
\sum_{k=1}^K \bar{X}_{Bk}\hat{\beta}_{Ak}
+
\sum_{k=1}^K \bar{X}_{Bk}\hat{\beta}_{Ak} \\
=&
\left[
(
\hat{\beta}_{B0} -\hat{\beta}_{A0}
)
+ 
\sum_{k=1}^K \bar{X}_{Bk}
(
\hat{\beta}_{Bk}
-
\hat{\beta}_{Ak}
)
\right]
+
\sum_{k=1}^K (
\bar{X}_{Bk}
- 
\bar{X}_{Ak} 
)
\hat{\beta}_{Ak} \\
=& 
\hat{\Delta}_S^\mu + \hat{\Delta}_X^\mu,
\end{align*}
where $\hat{\Delta}_S^\mu$ is the \emph{unexplained (structural) part}, and $\hat{\Delta}_X^\mu$ the \emph{explained} (by covariates $X$) part.
\item \emph{Aggregate decomposition} refers to decomposing $\hat{\Delta}_O^\mu  = \hat{\Delta}_S^\mu + \hat{\Delta}_X^\mu$, while \emph{detailed decomposition} refers to subdividing the effect for each $k=1, ..., K$
\item Take away messages from this chapter
\begin{enumerate}
\item The wage structure effect can be interpreted as a treatment effect.
\item Going beyond the mean is a ``solved'' problem for aggregate decomposition.
\item  Going beyond the mean is more difficult for the detailed decomposition.
\item The analogy between quantile and standard (mean) regressions is not helpful.
\item Decomposing proportions is easier than decomposing quantiles.
\item There is no general solution to the ``omitted'' group problem.
\end{enumerate}
\end{itemize}

\section{Identification}
\begin{assumption}{1}{Mutually Exclusive Groups}
The population of agents can be divided into two mutually exclusive groups: $A$ and $B$. Thus, for an agent $i$, $D_{Ai} + D_{Bi}=1$, where $D_{gi}=\ind \{i\text{ is in }g\}$.
\end{assumption}

\begin{itemize}
\item This rules out decomposing wage differentials between overlapping groups e.g. Blacks, Whites, and Hispanics, who can be Black or White. (In this case, dummy variable method is a more natural approach.)
\item Overall, this assumption is not very restrictive.
\end{itemize}

\subsection{Aggregate decomposition}
We would like to decompose the observable ``overall difference'' (e.g. wage gap) into ``four decomposition terms'' under ``overlapping support assumption'' and ``ignorability assumption''.
\subsubsection{Overall difference}
\begin{itemize}
\item Let the distributional statistic of interest be $v(F_{Y_g|D_s})$, where $v:F \rightarrow \R$. $F_{Y_g|D_s}$ can be both observed (when $g=s$) and counterfactual (when $g \neq s$) distributions.
\item The overall $v$-difference in wages between the two groups in terms of the distributional statistic $v$ is
\begin{equation}
\Delta^v_O = v(F_{Y_B|D_B}) - v(F_{Y_A|D_A})
\end{equation}
\begin{itemize}
\item Overall difference is generally not what we want to uncover. We are more interested in the counterfactuals, e.g. what if group $A$ workers' pay follow group $B$ workers' wage profile.
\end{itemize}
\end{itemize}

\begin{assumption}{2}{Structural Form}
A worker $i$ from group $g$ is paid according to the wage structure $m_g$, which are functions of worker $i$'s observable ($X$) and unobservable ($\epsilon$) characteristics.
\begin{equation}
Y_{Ai} = M_A(X_i, \epsilon) \ \text{and} \ Y_{Bi} = M_B(X_i, \epsilon)
\end{equation}
\end{assumption}

\begin{itemize}
\item This assumption implies there are only 3 sources of wage differentials
\begin{enumerate}
\item wage setting functions $m$
\item distribution of $X$
\item distribution of $\epsilon$
\end{enumerate}
\end{itemize}

\begin{assumption}{3}{Simple Counterfactual Treatment} A counterfactual wage structure for workers in group $B$, $m^c(\cdot, \cdot) \equiv m_A(\cdot, \cdot)$, and vice versa. 
\end{assumption}

\begin{itemize}
\item For example, counterfactual wage of a worker in group $B$ is $Y^C_{A|D_B, i}=m^C(X_i, \epsilon_i)=m_A(X_i, \epsilon_i)$.
\item This assumption rules out the existence of another counterfactual wage distribution e.g. what would happen to the pay structure of female in a world in which there is no gender.
\item This assumption makes decomposition framework a partial equilibrium framework.
\end{itemize}

\subsubsection{Four decomposition terms}
\begin{itemize}
\item Overall difference between two groups can be attributed to
\begin{enumerate}
\item Differences in returns to $X$
\item Differences in returns to $\epsilon$
\item Differences in the distribution of $X$
\item Differences in the distribution of $\epsilon$
\end{enumerate}

\item 1 and 2 can be summed up as structural difference, $\Delta^v_s$. This is also called ``$v$-wage structure effect.'' Then we have
\begin{equation}
\Delta^v_O = \Delta^v_s + \Delta^v_X + \Delta^v_\epsilon.
\end{equation} 
\end{itemize}

\subsubsection{Overlapping support}
\begin{assumption}{4}{Overlapping Support} Let the support of wage setting factors $[X', \epsilon']'$ be $\mathcal{X} \times \mathcal{E} $. For any $[x', e']'$ in $\mathcal{X} \times \mathcal{E} $, $0<\Pr[D_B=1|X=x, \epsilon = e]<1$.
\end{assumption}

\begin{itemize}
\item This assumption implies there is no $X=x$ or $\epsilon =e$ that divides the population into groups. A counter example is when only high-ability people are union workers and only low-ability people are non-union workers.
\item In the decomposition of gender wage differential, this assumption is often violated.
\end{itemize}

\subsubsection{Ignorability}
\begin{itemize}
\item $X$ may be correlated with $\epsilon$. 
\item For workers in group $B$, observed wage follows the following distribution
\begin{equation}
F_{Y_B|D_B}(y) = \int F_{Y_B|X,D_B}(y|X=x) \cdot dF_{X|D_B}(x)
\label{eq:actualwage}
\end{equation}
The counterfactual wage follows
\begin{equation}
F_{Y^C_A|X=X:D_B}(y) = \int F_{Y_A|X,D_A}(y|X=x) \cdot dF_{X|D_B}(x).
\label{eq:counterwage}
\end{equation}
That is, we integrate the counterfactual distribution over the distribution of observable characteristics of workers in group $B$\footnote{Note that we can easily replace $F_{Y_B|X,D_B}(y|X=x)$ with $F_{Y_A|X,D_A}(y|X=x)$ under simple counterfactual assumption.}.

\item Consider the RHS. For $g=A, B$, conditional distribution is defined as follows.
\begin{equation}
F_{Y_g|X,D_g}(y|X=x) = 
\Pr
(
m_g(X,\epsilon)
\leq
y
|
X=x, D_g=1
)
\label{eq:wagedist}
\end{equation}

\item According to \eqref{eq:wagedist}, the distribution of wages depends only on the distribution of $\epsilon$ and the wage structure $m_g(\cdot)$. Therefore, when we replace the conditional distribution on the RHS of \eqref{eq:actualwage} with the counterfactual one as in \eqref{eq:counterwage}, we are replacing both functional form and distribution of $\epsilon$. $\Rightarrow$ We need to separate the two from each other, i.e. ignorability assumption

\end{itemize}

\begin{assumption}{5}{Ignorability/Conditional independence}
For $g=A, B$, let $(D_g, X, \epsilon)$ have a joint distribution. For all $x$ in $\mathcal{X}$, $D_g \perp \epsilon|X$. 
\end{assumption}

\begin{itemize}
\item It implies that $\Delta^v_S = v(F_{Y_B|D_B})-v(F_{Y_A^C:X=X|D_B})$, i.e. difference between actual and counterfactual depends solely on wage structure (i.e. structural functions $m_B(\cdot,\cdot)$ and $m_A(\cdot,\cdot)$).
\item This assumption is often called ``unconfoundedness'' or ``selection on observables'' in the program evaluation literature.
\item It is somewhat strong assumption. There are 3 cases under which it may not hold.
\begin{enumerate}
\item Differential selection into labor marker \\
The decision to participate may be different for men and women.
\item Self-selection into groups A and B based on unobservables.
\item Choice of $X$ and $\epsilon$.
\end{enumerate}
\end{itemize}

\begin{prop}{1}{Identification of the Aggregate Decomposition}
Under simple counterfactual, overlapping support, and ignorability, the overall $v$-gap can be written as
$$
\Delta^v_O = \Delta^v_s + \Delta^v_X,
$$
where
\begin{enumerate}
\item $\Delta^v_S = v(F_{Y_B|D_B})-v(F_{Y_A^C:X=X|D_B})$, i.e. wage structure term solely reflects the difference between the structural functions $m_B(\cdot,\cdot)$ and $m_A(\cdot,\cdot)$.
\item $\Delta^v_X = v(F_{Y_A^C:X=X|D_B}) - v(F_{Y_A|D_A})$, i.e. composition effect term solely reflects the difference between the $X$ and $\epsilon$ between 2 groups.
\end{enumerate}
\end{prop}
\begin{itemize}
\item Explanation:
\begin{itemize}
\item $v(F_{Y_B|D_B})-v(F_{Y_A^C:X=X|D_B})$ is the difference of two distributions that have different shapes but the same $X$s. 
\item $v(F_{Y_A^C:X=X|D_B}) - v(F_{Y_A|D_A})$ is the difference of two distributions that have the same shape but different $X$s. 
\end{itemize}

\item Note that 
\begin{align*}
\Delta^v_O 
=& v(F_{Y_B|D_B}) - v(F_{Y_A|D_A}) \\
=& v(F_{Y_B|D_B})-v(F_{Y_A^C:X=X|D_B}) + v(F_{Y_A^C:X=X|D_B}) - v(F_{Y_A|D_A}) \\
=&
\Delta^v_S + \Delta^v_X \\
\end{align*}
\end{itemize}

\begin{assumption}{6}{Invariance of Conditional Distributions}
When replace the conditional distribution of \eqref{eq:actualwage} as in \eqref{eq:counterwage}, $F_{Y_A|X,D_A}(y|X=x)$ remains valid for any $x \in \mathcal{X}$
\end{assumption}

\begin{itemize}
\item This assumption is not empirically testable. Thus, it must be justified using the economic context.
\end{itemize}

\bibliographystyle{apacite}
\bibliography{} 


\end{document}
